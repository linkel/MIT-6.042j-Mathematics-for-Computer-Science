\documentclass{article}
\usepackage[utf8]{inputenc}
\usepackage{amsmath}

\title{MIT 6.042J HW1 2005}
\author{Kelly Lin }
\date{June 2019}

\begin{document}

\maketitle

\section{Problem 1}

Prove that $\sqrt[3]{2}$ is not sensible. 
\\
I will be doing a proof by contradiction. 
\\
Let us assume that $\sqrt[3]{2}$ is sensible. We will set it equal to $\sqrt{\frac{a}{b}}$.
\\
We square both sides, then cube both sides to get
\begin {align*}
\frac{a}{b} = \sqrt[3]{4}
\end {align*}
This implies that $\sqrt[3]{4}$ is a rational number. Rational numbers can be written as a fraction in lowest terms where the numerator and denominator are both integers. 
\begin {align*}
\\
\frac{x^{3}}{y^{3}} = 4
\\
x^{3} = 4y^{3}
\end {align*}
$4y^{3}$ must be even because 4, an even number, multiplied by anything turns it even.
\\
$x^{3}$ must be even because the right side was even. 
\\
x itself must be even because of that. Then $x^{3}$ must be divisible by 8. 
\\
That means that $4y^{3}$ must also be divisible by 8. 
\\
Since the 4 isn't enough to cover 8, $y^{3}$ must also be divisible by 2, therefore even. 
\\
Then y must be even. 
\\ 
If both x and y are even then they are not in lowest terms. 
\\
This contradiction means that $\sqrt[3]{4}$ isn't rational. 
\\
Since it being sensible implied that it was rational, it can't be sensible. 

\\

\section{Problem 2}
"There is a student who has emailed exactly two other people in the class, besides possibly herself."
Okay, so we know there's three students here. I'll call them s, t, and u. 
\\
\begin {align*}
\exists s \exists t \exists u \in S 
\end {align*}
\\
s has emailed t and s has emailed u.
\\
\begin {align*}
E(s,t) \wedge E(s,u)
\end {align*}
\\
s is a different student from u, and a different student from t, and same goes for u and t. 
\\
\begin {align*}
s \neq u \wedge s \neq t \wedge u \neq t
\end {align*}
\\
Of the students that s has emailed, those students are either u, t, or themselves. So we can say "Every student o that s has emailed, o is equal to s, t, or u."
\begin {align*}
\forall o \in S E(s, o) \rightarrow o = s \vee o = t \vee o = u
\end {align*}
So we can AND all of the phrases together to make the predicate. 
\begin {align*}
\exists s \exists t \exists u \in S. E(s,t) \wedge E(s,u) \wedge s \neq u \wedge s \neq t \wedge u \neq t \wedge \forall o \in S. E(s, o) \rightarrow o = s \vee o = t \vee o = u
\end {align*}

\section{Problem 3}

Write in predicate form. Addition, multiplication, equality, but no constants. Prove stuff first before using. This is over the domain of the natural numbers (non-negative integers).
\\\\
a. n is the sum of three perfect squares \\
\begin {align*}
\exists x \exists y \exists z. x*x + y*y + z*z = n
\end {align*}
\\
I don't have to specify anything special on these since it's fine if x, y, and z are the same as each other. 
\\\\
b. x $>$ 1
\\
Since we can't use constants, we need to first define $x = 1$. 
\\
\begin {align*}
\forall y. (y*x = y)
\end {align*}
\\
Now we can set things equal to one using the above definition. 
\\
\begin {align*}
\exists y. (x > y) \wedge (y = 1)
\end {align*}
\\
Kind of confused on who takes the exists in the above line. 
\\\\
c. n is a prime number.
\\\\
Well, I know that it isn't possible to have two variables that multiply each other to become n. 
\\
\begin {align*}
\neg(x*y = n)
\end {align*}
\\
Also have to declare it. So it's more like...
\\
\begin {align*}
\neg (\exists x \exists y. (x > 1 \wedge y > 1 \wedge x*y = n))
\end {align*}
\\
And we know that n, the prime number, has to be bigger than 1. So altogether:
\\
\begin {align*}
IS-PRIME(n) ::= (n > 1) \wedge \neg (\exists x \exists y. (x > 1 \wedge y > 1 \wedge x*y = n))
\end {align*}
\\\\
d. n is a product of two distinct primes.
\begin {align*}
\exists a \exists b. IS-PRIME(a) \wedge IS-PRIME(b) \wedge \neg (a = b) \wedge (a * b = n)
\end {align*}
\\\\
e. There is no largest prime number.
\\\\
So we're going to want to do a NOT on the statement "There is a largest prime number." To say that there is a largest prime number, we want to specify that 1. there exists a p where IS-PRIME(p) and 2. for all q where q is a prime number, that it follows that there is a p bigger than or equal to q. So let's see about the first statement...
\begin {align*}
\exists p. IS-PRIME(P)
\end {align*}
And the second statement:
\begin {align*}
\forall q. IS-PRIME(q) \rightarrow p \leq q
\end {align*}
And then we want to AND those two together and then NOT the whole thing.
\begin {align*}
\neg (\exists p. IS-PRIME(P) \wedge (\forall q. IS-PRIME(q) \rightarrow p \leq q))
\end {align*}
\\\\
f. Goldbach conjecture. Every even natural number $n > 2$ can be expressed as the sum of two primes. 
\\\\
This is a combo of a couple of statements. We need to explain what an even number is. We need to explain what it means for something to be bigger than 2. This requires us to define how something can be equal to 1 (which we did do back up in problem 3b). Then we can make use of the IS-PRIME function and link it all together with a right arrow "implies". 
\\
This is the definition of 1:
\begin {align*}
\forall y. (y*m = y)
\end {align*}
This is the definition of a thing to be bigger than 2:
\begin {align*}
\exists y. (y = 1) \wedge (x > y + y)
\end {align*}
This is the definition of an even number:
\begin {align*}
\forall y. (x = 2y)
\end {align*}
So pulling it together:
\begin {align*}
\forall n. ((n > 2 \wedge \exists y. n = 2y) \rightarrow \exists p \exists q. IS-PRIME(p) \wedge IS-PRIME(q) \wedge (n = p + q))
\end {align*}
\\
g. Bertrand's Postulate. If $n > 1$ then there is always at least one prime p such that $n < p < 2n$. 
\\\\
Since we have already shown how to have a variable be less than another variable, how a thing can be prime, we can simply write:
\begin {align*}
\forall n. (n > 1) \rightarrow \exists p. IS-PRIME(p) \wedge (n < p) \wedge (p < 2n)
\end {align*}
\\\\

\section{Problem 4}
\\
I will be doing a proof by contradiction.
\\
So a surjection is a function where every single element of the y result (codomain) is mapped to by at least one element of the x input (domain).
\\
Let's say there is a surjective function (aka surjection) $f: A \rightarrow P(A)$, where P is the powerset.  
\\
Making use of that helpful hint, let's define a set S where:
\begin {align*}
S ::= x \in A | x \not \in f(x)
\end {align*}
Looking at that first part of the definition, this means that if an x belongs to S, then it must not be in f(x):
\\
\begin {align*}
(x \in S) \leftrightarrow (x \not \in f(x))
\end {align*}
Because x is an element that belongs to set A, by definition S must be a subset of A. If $S \subseteq A$, then S must also be in the powerset of A, P(A). So S must equal f(a) for an $a \in A$.
\begin {align*}
(x \in f(a)) \leftrightarrow (x \not \in f(x))
\end {align*}
If we plug in x for a in the above equation, it's a contradiction. 
\begin {align*}
(x \in f(x)) \leftrightarrow (x \not \in f(x))
\end {align*}
Can't be both in it and not in it at the same time. Therefore if A is infinite, there still does not exist a surjection f from A to its powerset P(A). 
\section{Problem 5}

\section{Problem 6}


\end{document}

