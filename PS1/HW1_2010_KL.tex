\documentclass{article}
\usepackage[utf8]{inputenc}
\usepackage{amsmath}

\title{MIT 6.042J HW1 2010}
\author{Kelly Lin }
\date{June 2019}

\begin{document}

\maketitle

\section{Problem 1}

\section{Problem 2}
a. Use a truth table to prove or disprove the following:
\begin {align*}
\neg (P \vee (Q \wedge R)) = (\neg P) \wedge (\neg Q \vee \neg R)
\end {align*}
I made a truth table on scratch paper--not interested in reproducing it on here. 
This equation is true. There are 5 cases that are false and 3 that are true and they match up. 
\\\\
b.
\begin {align*}
\neg (P \wedge (Q \vee R)) = \neg P \vee (\neg Q \vee \neg R)
\end {align*}
This equation is false. The LHS has three false and 5 true. The RHS had 1 false and 7 true. 
\section{Problem 3}
Using nand, find an equivalent expression.
\\
i. $A \wedge B$
\\
$\neg (A \; nand \; B)$
\\\\
ii. $A \vee B$
\\
$(\neg A \; nand \; \neg B)$
\\
I actually came up with the nand version of it before I did the not version.
\\\\
iii. $A \rightarrow B$
\\
$A \; nand \; (\neg B)$
\\
For this one, we want something that will be true except when A is false and B is true. Because for all the other combos of A and B, they are valid. However when A is false and B is true, A implies B so if A is true and B is false that's not implying and the statement doesn't hold.

\end{document}

